\documentclass[12pt,a4paper,oneside]{article}
\usepackage[utf8]{inputenc}
\usepackage[english,russian]{babel}
\usepackage[left=2cm, right=2cm, top=2cm, bottom=2cm]{geometry}
\usepackage{enumitem}
\usepackage{amsfonts}
\usepackage{amsmath}

\begin{document}

\section*{Задание №2. Теоремы об исчислении высказываний. Знакомство с интуиционистским исчислением высказываний.}
\begin{enumerate}

\item Только для очной практики, сорян :(
\item Обозначим $A = \alpha \rightarrow (\beta \rightarrow  \gamma), 
B = (\alpha \rightarrow \beta) \rightarrow  \gamma$ \\
По Теореме о полноте исчисления высказываний если $ \models A \rightarrow B $, то 
$ \vdash A \rightarrow B $ и если $ \models B \rightarrow A $, то 
$ \vdash B \rightarrow A $ \\
Построим таблицу истинности:\\
\begin{center}
\begin{tabular}{ c c c | c c c c c c }
 $ \alpha $ & $ \beta $ & $ \gamma $ & $ \beta \rightarrow  \gamma $ & $ \alpha \rightarrow \beta $ & $ \alpha \rightarrow (\beta \rightarrow  \gamma) $ & $ (\alpha \rightarrow \beta) \rightarrow  \gamma $ & $ A \rightarrow B$ & $ B \rightarrow A $ \\ 
 \hline
 Л & Л & Л & И & И & И & Л & Л & И\\
Л & Л & И & И & И & И & И & И & И\\
Л & И & Л & Л & И & И & Л & Л & И\\
Л & И & И & И & И & И & И & И & И\\
И & Л & Л & И & Л & И & И & И & И\\
И & Л & И & И & Л & И & И & И & И\\
И & И & Л & Л & И & Л & Л & И & И\\
И & И & И & И & И & И & И & И & И
\end{tabular}
\end{center}

Получили, что $ \models B \rightarrow A $, тогда $ \vdash B \rightarrow A $, то есть 
$ \vdash ((\alpha \rightarrow \beta) \rightarrow  \gamma) \rightarrow (\alpha \rightarrow (\beta \rightarrow  \gamma))$

\item Доказать, что если $ \Gamma \models \alpha $, то $ \Gamma \vdash \alpha $
\item Доказать, что если $ \Gamma \vdash \alpha $, то $ \Gamma  \models \alpha $

\end{enumerate}

\end{document}